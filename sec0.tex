\subsection{Basics}
The subset of a given set $A$ is
\begin{align*}
    B = \{a\in A\ |\ \dots\text{(conditions on a)}\dots\}
\end{align*}
The {\sl order} (or {\sl cardinality}) of a set $A$ will be denoted by $\abs{A}$.
If $A$ is a finite set, the order of $A$ is simply the number of elements of $A$.

The {\sl Cartesian product} of two sets $A$ and $B$ is the collation $A\times B = \{(a,b)\ |\ a\in A,b\in B\}$,
of ordered pairs of elements from $A$ and $B$.

The following notation for some common sets of numbers
\begin{enumerate}[leftmargin=20pt, itemsep=0pt, topsep=3pt]
    \item {\bf Integers}: $\mb{Z} = \{0,\pm 1, \pm 2,\dots\}$.
    \item {\bf Rational numbers}: $\mb{Q} = \{a/b\ |\ a,b\in\mb{Z},b\neq 0\}$.
    \item {\bf Real numbers}: $\mb{R} = \{\text{all decimal expansions} \pm d_{1}d_{2}\dots d_{n}.a_{1}a_{2}a_{3}\dots\}$.
    \item {\bf Complex numbers}: $\mb{C} = \{a+bi\ |\ a,b\in\mb{R},i^{2}=-1\}$.
    \item $\mb{Z}^{+}$, $\mb{Q}^{+}$ and $\mb{R}^{+}$ will denote the positive (nonzero) elements in $\mb{Z}$, $\mb{Q}$ and $\mb{R}$, respectively.
\end{enumerate}
The notation $f:A\to B$ or $A\xrightarrow{\,f\,} B$ to denote a {\sl function} (or {\sl map}) $f$ from $A$ to $B$,
and the value of $f$ at $a$ is denoted by $f(a)$.
The set $A$ is called the {\sl domain} of $f$ and
$B$ is called the {\sl codomain} of $f$.

The notation $f:a\mapsto b$ or $a\mapsto b$ if $f$ is understood indicates that $f(a)=b$, i.e., the function is being specified on {\sl elements}.
If the function $f$ is not specified on elements, it is important in general to check
that $f$ is {\sl well defined}, i.e., is unambiguously determined.

The set
\begin{align*}
    f(A) = \{b\in B\ |\ b=f(a)\text{, for some }a\in A\}
\end{align*}
is a subset of $B$, called the {\sl range} or {\sl image} of $f$ (or the {\sl image} of $A$ under $f$).

For each subset $C$ of $B$, the set
\begin{align*}
    f^{-1}(C) = \{a\in A\ |\ f(a)\in C\}
\end{align*}
consisting of the elements of $A$ mapping into $C$ under $f$ is called
the {\sl preimage} or {\sl inverse image} of $C$ under $f$.
For each $b\in B$, the preimage of $\{b\}$ under $f$ is called the {\sl fiber} of $f$ over $b$.
Note that $f^{-1}$ is not in general a function and that the fibers of $f$ generally
contain many elements since there may be many elements of $A$ mapping to the element $b$.

If $f:A\to B$ and $g:B\to C$, then the composite map $g\circ f:A\to C$ is defined by
\begin{align*}
    (g\circ f)(a) = g(f(a))
\end{align*}
\newpage
\thispagestyle{evenpagestyle}
\Definition{0.1-1}
{Let $f:A\to B$
\begin{enumerate}[leftmargin=20pt, itemsep=0pt, topsep=3pt]
    \item $f$ is {\sl injective} or is an {\sl injection} if whenever $a_{1}\neq a_{2}$, then $f(a_{1})\neq f(a_{2})$.
    \item $f$ is {\sl surjective} or is a {\sl surjection} if for all $b\in B$, there is some $a\in A$ such that $f(a)=b$.
    \item $f$ is {\sl bijective} or is a {\sl bijection} if it is both injective and surjective.
    If such a bijection $f$ exists from $A$ to $B$, we say $A$ and $B$ are in {\sl bijective correspondence}.
    \item $f$ has a {\sl left inverse} if there is a function $g:B\to A$ such that $g\circ f: A\to A$ is the identity map on $A$, i.e., $(g\circ f)(a)=a$ for all $a\in A$.
    \item $f$ has a {\sl right inverse} if there is a function $h:B\to A$ such that $f\circ h: B\to B$ is the identity map on $B$, i.e., $(f\circ h)(b)=b$ for all $b\in B$.
\end{enumerate}}

\Proposition{0.1-1}
{Let $f:A\to B$
\begin{enumerate}[leftmargin=20pt, itemsep=0pt, topsep=3pt]
    \item The map $f$ is injective if and only if $f$ has a left inverse.
    \item The map $f$ is surjective if and only if $f$ has a right inverse.
    \item The map $f$ is a bijection if and only if there exists $g:B\to A$ such 
    that $f \circ g$ is the identity map on $B$ and $g \circ f$ is the identity map on $A$.
    \item If $A$ and $B$ are finite sets with the same number of elements (i.e., $\abs{A} = \abs{B}$),
    then $f : A \to B$ is bijective if and only if $f$ is injective if and only if $f$ is surjective.
\end{enumerate}}
{\begin{enumerate}[leftmargin=20pt, itemsep=0pt, topsep=3pt]
    \item ($\Rightarrow$) Suppose $f$ is injective. Notice that if $b\in f(A)$ then there is a unique $a\in A$ such that
    
    $f(a)=b$. Choose any $a_{0}\in A$, and define $g:B\to A$ by
    \linespread{1}\selectfont
    \begin{align*}
        g(b) = \begin{cases}
            a & \text{if } b\in f(A) \\
            a_{0} & \text{if } b\notin f(A)
        \end{cases}
    \end{align*}
    \linespread{1.2}\selectfont
    Then $(g\circ f)(a) = a$ for all $a\in A$, so $g$ is a left inverse of $f$.
    
    ($\Leftarrow$) Suppose $f$ has a left inverse $g$, and that $f(a) = f(b)$.
    Then $g(f(a)) = g(f(b))$, and since $g\circ f:A\to A$, we have $a=b$, which shows $f$ is injective.
    \item ($\Rightarrow$) Suppose $f$ is surjective. Then every $b\in B$ is in the image of $f$, so for each $b\in B$
    pick an element $g(b)\in A$ such that $f(g(b))=b$. Then $g$ is a right inverse of $f$.

    ($\Leftarrow$) Suppose $f$ has a right inverse $g$ and let $b\in B$. Then $f(g(b))=b$ as $f\circ g: B\to B$.
    This shows $b\in f(A)$, so $f(A) = B$ and $f$ is surjective.
    \item ($\Rightarrow$) Suppose $f$ is a bijection, then $f$ is injective and surjective by definition.
    By part 1. there exists a left inverse $g:B\to A$ such that $g\circ f:A\to A$,
    and by part 2. there exists a right inverse $g:B\to A$ such that $f\circ g:B\to B$.
    
    ($\Leftarrow$) Suppose there exists $g:B\to A$ such that $f\circ g:B\to B$ and $g\circ f:A\to A$.
    Then by part 1., $f$ is surjective, and by part 2., $f$ is injective.
    Then $f$ is a bijection.
\end{enumerate}
}
\newpage
\thispagestyle{oddpagestyle}
\Proppf
{\begin{enumerate}[leftmargin=20pt, itemsep=0pt, topsep=0pt]
    \setcounter{enumi}{3}
    \item {\bf Claim}:
    \begin{itemize}[leftmargin=20pt, itemsep=0pt, topsep=3pt]
        \item [(1)] If $f:A \to B$ is injective, then $\abs{A} \leq \abs{B}$.
        \item [(2)] If $f:A \to B$ is surjective, then $\abs{A} \geq \abs{B}$.
        \item [(3)] If $f:A \to B$ is a bijection, then $\abs{A} = \abs{B}$.
    \end{itemize}
    {\it proof}. Let $A=\{a_{1},a_{2},\dots,a_{m}\}$ has $m$ elements.
    \begin{itemize}[leftmargin=20pt, itemsep=0pt, topsep=3pt]
        \item [(1)] $\{f(a_{1}),f(a_{2}),\dots,f(a_{m})\}$ is a subset of $B$, because $f$ is injective, $\abs{A} = m \leq \abs{B}$.
        \item [(2)] $\{f(a_{1}), f(a_{2}),\dots,f(a_{m})\}=B$ has at most $m$ different elements because $f$ is surjective, $\abs{A} = m \geq \abs{B}$.
        \item [(3)] This follows from (1) and (2), since $\abs{A} \leq \abs{B}$ and $\abs{A} \geq \abs{B}$, we have $\abs{A} = \abs{B}$.
    \end{itemize}
\end{enumerate}
}
The situation of part 3. of \textbf{\textsf{\color{orange!20!brown} Proposition 0.1-1}}, 
the map $g$ is necessarily unique and we shall say $g$ is the {\sl 2-sided inverse} (or {\sl inverse}) of $f$.

A {\sl permutation} of a set $A$ is simply a bijection from $A$ to itself.

If $A\subseteq B$ and $f:B\to C$, we denote the {\sl restriction} of $f$ to $A$ by $f|_{A}$.

If $A\subseteq B$ and $g:A\to C$ and there is a function $f:B\to C$ such that $f|_{A}=g$, we shall say $f$ is an {\sl extension} of $g$ to $B$.
\Definition{0.1-2}
{Let $A$ be a nonempty set.
\begin{enumerate}[leftmargin=20pt, itemsep=0pt, topsep=3pt]
    \item A {\sl binary relation} on a set $A$ is a subset $R$ of $A\times A$ and we write $a\sim b$ if $(a,b)\in R$.
    \item The relation $\sim$ on $A$ is said to be:
    \begin{enumerate}[leftmargin=20pt, itemsep=0pt, topsep=0pt, label=(\alph*)]
        \item {\sl reflexive} if $a\sim a$, for all $a\in A$
        \item {\sl symmetric} if $a\sim b$ implies $b\sim a$ for all $a,b\in A$
        \item {\sl transitive} if $a\sim b$ and $b\sim c$ implies $a\sim c$ for all $a,b,c\in A$
    \end{enumerate}
    A relation is an {\sl equivalence relation} if it is reflexive, symmetric and transitive.
    \item If $\sim$ defines an equivalence relation on $A$, then the {\sl equivalence class} of $a\in A$ is defined
    to be $\{x\in A\ |\ x\sim a\}$. Elements of the equivalence class of $a$ are said to be {\sl equivalent} to $a$.
    If $C$ is an equivalence class, any element of $C$ is called a {\sl representative} of the class $C$.
    \item A {\sl partition} of $A$ is any collection $\{A_{i}\ |\ i\in I\}$ of nonempty subsets of $A$ ($I$ some indexing set)
    such that
    \begin{enumerate}[leftmargin=20pt, itemsep=0pt, topsep=0pt, label=(\alph*)]
        \item $A = \cup_{i\in I}A_{i}$
        \item $A_{i} \cap A_{j} = \varnothing$ for all $i,j\in I$ with $i\neq j$, i.e.,
        $A$ is the disjoint union of the sets in the partition.
    \end{enumerate}
\end{enumerate}
}
\newpage
\thispagestyle{evenpagestyle}
\Proposition{0.1-2}
{Let $A$ be a nonempty set.
\begin{enumerate}[leftmargin=20pt, itemsep=0pt, topsep=3pt]
    \item If $\sim$ defines an equivalence relation on $A$, then the set of equivalence classes of $\sim$ form a partition of $A$.
    \item If $\{A_{i}\ |\ i\in I\}$ is a partition of $A$, then there is an equivalence relation on $A$ whose equivalence classes are precisely the sets $A_{i}$, $i\in I$.
\end{enumerate}}
{
\href{https://math.libretexts.org/Courses/Monroe\_Community\_College/MTH\_220\_Discrete\_Math/6\%3A\_Relations/6.3\%3A\_Equivalence\_Relations\_and\_Partitions}
{Link}
}

\subsection{Properties of the integers}
\subsubsection{Well Ordering of \texorpdfstring{$\mb{Z}$}{Z}}
If $A$ is any nonempty subset of $\mb{Z}^{+}$, there is some element $m\in A$ such that
$m\le a$, for all $a\in A$ ($m$ is call a {\sl minimal element} of $A$).

\subsubsection{Divides}
If $a,b\in \mb{Z}$ with $a\neq 0$, we say $a$ {\sl divides} $b$ if there is an element $c\in\mb{Z}$ such that
$b = ac$. In this case, we write $a\mid b$; if $a$ does not divide $b$, we write $a\nmid b$.

\subsubsection{Greatest Common Divisor (g.c.d.)}
If $a,b\in \mb{Z}\,\diagdown\{0\}$, there is a unique positive integer $d$, called the {\sl greatest common divisor} of $a$ and $b$
(or g.c.d. of $a$ and $b$), satisfying:
\begin{enumerate}[leftmargin=20pt, itemsep=0pt, topsep=3pt, label=(\arabic*)]
    \item $d\mid a$ and $d\mid b$ ($d$ is a common divisor of $a$ and $b$)
    \item If $e\mid a$ and $e\mid b$, then $e\le d$ ($d$ is the greatest such divisor)
\end{enumerate}
The g.c.d. of $a$ and $b$ will be denoted by $(a,b)$ (or $\gcd(a,b)$).
If $(a,b)=1$, we say that $a$ and $b$ are {\sl relatively prime}.

\subsubsection{Least Common Multiple (l.c.m.)}
If $a,b\in \mb{Z}\,\diagdown\{0\}$, there is a unique positive integer $l$, called the {\sl least common multiple} of $a$ and $b$
(or l.c.m. of $a$ and $b$), satisfying:
\begin{enumerate}[leftmargin=20pt, itemsep=0pt, topsep=3pt, label=(\arabic*)]
    \item $a\mid l$ and $b\mid l$ ($l$ is a common multiple of $a$ and $b$)
    \item If $a\mid m$ and $b\mid m$, then $l\le m$ ($l$ is the least such multiple)
\end{enumerate}
The l.c.m. of $a$ and $b$ will be denoted by $[a,b]$ (or $\mathrm{lcm}(a,b)$).
The connection between the g.c.d. $d$ and the l.c.m. $l$ of two integers $a$ and $b$ is given by $dl=ab$.

\subsubsection{The Division Algorithm}
If $a,b\in \mb{Z}\,\diagdown\{0\}$, then there exist unique $q,r\in \mb{Z}$ such that
\begin{align*}
    a = qb + r \text{\qquad and \qquad} 0\le r < \abs{b}
\end{align*}
where $q$ is the {\sl quotient} and $r$ is the {\sl remainder}.
\newpage
\thispagestyle{oddpagestyle}

\subsubsection{The Euclidean Algorithm}
If $a,b\in \mb{Z}\,\diagdown\{0\}$, then we obtain a sequence of quotients and remainders
\begin{align*}
    a &= q_{0} b + r_{0}\tag{0}\\
    b &= q_{1} r_{0} + r_{1}\tag{1}\\
    r_{0} &= q_{2} r_{1} + r_{2}\tag{2}\\
    r_{1} &= q_{3} r_{2} + r_{3}\tag{3}\\
    &\vdots\\
    r_{n-2} &= q_{n} r_{n-1} + r_{n}\tag{n}\\
    r_{n-1} &= q_{n+1} r_{n}\tag{$n+1$}
\end{align*}
where $r_{n}$ is the last nonzero reminder. Such an $r_{n}$ exists since
$\abs{b}>\abs{r_{0}}>\abs{r_{1}}>\cdots>\abs{r_{n}}$ is a decreasing sequence of strictly positive integers
if the reminders are nonzero and such a sequence cannot continue indefinitely.
Then $r_{n}$ is the g.c.d. $(a,b)$ of $a$ and $b$.
\Example{0.2.6-1}
{Find the g.c.d. of $a=57970$ and $b=10353$.}
{Applying the Euclidean algorithm, we have
\begin{align*}
    57970 &= (5) 10353 + 6205\\
    10353 &= (1) 6205 + 4148\\
    6205 &= (1) 4148 + 2057\\
    4148 &= (2) 2057 + 34\\
    2057 &= (60) 34 + 17\\
    34 &= (2) 17
\end{align*}
Thus, the g.c.d. of $57970$ and $10353$ is $(57970, 10353) = 17$.
}

\subsubsection{\texorpdfstring{$\mb{Z}\,$}{Z}-linear Combinations}
One consequence of the Euclidean Algorithm which we shall use regularly
is the following: if $a,b\in \mb{Z}\,\diagdown\{0\}$, then there exist $x,y\in\mb{Z}$ such that
\begin{align*}
    (a,b) = ax + by
\end{align*}
that is, {\sl the g.c.d. of $a$ and $b$ is a $\mb{Z}\,$-linear combination of $a$ and $b$}.
This follows by recursively writting the element $r_{n}$ in the Euclidean Algorithm in terms of the previous
remainders (namely, use equation (n) above to solve for $r_{n} = r_{n-2} - q_{n}r_{n-1}$
in terms of the remainders $r_{n-1}$ and $r_{n-2}$, then use equation ($n-1$) to write $r_{n}$
in terms of the remainders $r_{n-2}$ and $r_{n-3}$, etc., eventually writing $r_{n}$ in terms of $a$ and $b$).
\newpage
\thispagestyle{evenpagestyle}
\Example{0.2.7-1}
{Use the Euclidean Algorithm to find integers $x,y$ such that
\begin{align*}
    (57970, 10353) = 57970x + 10353y
\end{align*}
}
{Based on \textbf{\textsf{\color{blue!60!black} Example 0.2.6-1}}
we know that $(57970, 10353) = 17$. Start from the fifth equation in the Euclidean Algorithm,
\begin{align*}
    17 &= 2057 - 60\cdot 34\\
    &= 2057 - 60\cdot (4148 - (2) 2057) = 121\cdot 2057 - 60\cdot 4148\\
    &= 121\cdot (6205 - (1) 4148) - 60\cdot 4148 = 121\cdot 6205 - 181\cdot 4148\\
    &= 121\cdot 6205 - 181\cdot (10353 - (1) 6205) = 302\cdot 6205 - 181\cdot 10353\\
    &= 302\cdot (57970 - (5) 10353) - 181\cdot 10353\\
    &= 302\cdot 57970 + (-1691)\cdot 10353
\end{align*}
Thus, $x=302$ and $y=-1691$ is a solution of $(57970, 10353) = 57970x + 10353y$.
}

\subsubsection{Prime and Composite Numbers}
An element $p$ of $\mb{Z}^{+}$ is called a {\sl prime} if $p>1$ and the only positive divisors of $p$ are $1$ and $p$.
An integer $n>1$ which is not prime is called {\sl composite}.

\subsubsection{The Fundamental Theorem of Arithmetic}
If $n\in\mb{Z}$, $n>1$, then $n$ can be factored uniquely into the product of primes, i.e., there are distinct primes
$p_{1}, p_{2}, \dots, p_{s}$ and positive integers $\alpha_{1}, \alpha_{2},\dots,\alpha_{s}$ such that
\begin{align*}
    n = p_{1}^{\alpha_{1}}p_{2}^{\alpha_{2}}\dots p_{s}^{\alpha_{s}}
\end{align*}
This factorization is unique in the sense that if $q_{1},q_{2},\dots,q_{t}$ are any distinct
primes and positive integers $\beta_{1},\beta_{2},\dots,\beta_{t}$ such that
\begin{align*}
    n = q_{1}^{\beta_{1}}q_{2}^{\beta_{2}}\dots q_{t}^{\beta_{t}}
\end{align*}
then $s=t$ and if we arrange the two sets of primes in increasing order, then $q_{i} = p_{i}$
and $\alpha_{i} = \beta_{i}$ $1\le i\le s$.

Suppose the positive integers $a$ and $b$ are expressed as products of prime powers:
\begin{align*}
    a = p_{1}^{\alpha_{1}}p_{2}^{\alpha_{2}}\dots p_{s}^{\alpha_{s}} \qquad b = p_{1}^{\beta_{1}}p_{2}^{\beta_{2}}\dots p_{s}^{\beta_{s}}
\end{align*}
where $p_{1}, p_{2}, \dots, p_{s}$ are distinct and the exponents are $\ge 0$
(we allow the exponents to be $0$ here, so that the products are taken over the same set of primes -- the exponent will be $0$ if that prime is not actually a divisor).
Then the g.c.d. of $a$ and $b$ is
\begin{align*}
    (a,b) = p_{1}^{\min\{\alpha_{1}, \beta_{1}\}}p_{2}^{\min\{\alpha_{2}, \beta_{2}\}}\dots p_{s}^{\min\{\alpha_{s}, \beta_{s}\}}
\end{align*}
and the l.c.m. of $a$ and $b$ is
\begin{align*}
    [a,b] = p_{1}^{\max\{\alpha_{1}, \beta_{1}\}}p_{2}^{\max\{\alpha_{2}, \beta_{2}\}}\dots p_{s}^{\max\{\alpha_{s}, \beta_{s}\}}
\end{align*}
\newpage
\thispagestyle{oddpagestyle}

\subsubsection{Euler \texorpdfstring{$\varphi\,$}{phi}-function}
For $n\in \mb{Z}^{+}$, let $\varphi(n)$ be the number of positive integers $a\le n$
with $a$ relatively prime to $n$, i.e., $(a,n) = 1$. For example, $\varphi(12) = 4$
since the positive integers $1, 5, 7, 11$ are the only positive integers less than or equal to $12$
which have no factors in common with $12$. For prime $p$, $\varphi(p) = p-1$, and more generally,
for all $a\ge 1$ we have the formula
\begin{align*}
    \varphi(p^{a}) = p^{a} - p^{a-1} = p^{a-1}(p-1)
\end{align*}
The function $\varphi$ is {\sl multiplicative} in the sense that
\begin{align*}
    \varphi(ab) = \varphi(a)\varphi(b)\qquad \text{if } (a,b) = 1
\end{align*}
(note that it is important here that $a$ and $b$ be relatively prime).
Together with the formula above this gives a general formula for the values of $\varphi$ : 
if $n=p_{1}^{\alpha_{1}}p_{2}^{\alpha_{2}}\dots p_{s}^{\alpha_{s}}$, then
\begin{align*}
    \varphi(n) &= \varphi(p_{1}^{\alpha_{1}})\varphi(p_{2}^{\alpha_{2}})\dots\varphi(p_{s}^{\alpha_{s}})\\
    &= p_{1}^{\alpha_{1}-1}(p_{1} - 1) p_{2}^{\alpha_{2}-1}(p_{2} - 1)\dots p_{s}^{\alpha_{s}-1}(p_{s} - 1)
\end{align*}
\Example{0.2.10-1}
{Find the value of $\varphi(36)$.
}
{
The prime factorization of $36$ is $36 = 2^{2}\cdot 3^{2}$, therefore
\begin{align*}
    \varphi(36) &= \varphi(2^{2})\varphi(3^{2})\\
    &= 2^{2-1}(2-1)3^{2-1}(3-1)\\
    &= 2\cdot 1\cdot 3\cdot 2 = 12
\end{align*}
}

\subsection{\texorpdfstring{$\mb{Z}/n\mb{Z}$}{Z/nZ} : The integers modulo \texorpdfstring{$n$}{n}}
Let $n$ be a fixed positive integer. Define a relation on $\mb{Z}$ by
\begin{align*}
    a\sim b\quad\text{if and only if}\quad n\mid (a-b)
\end{align*}
Clearly $a\sim a$, and $a\sim b$ implies $b\sim a$ for any integers $a$ and $b$, so this relation
is trivially reflexive and symmetric. If $a\sim b$ and $b\sim c$, then $n$ divides $a-b$ and $n$ divides $b-c$,
so $n$ also divides the sum of these two integers, i.e., $n$ divides $(a-b)+(b-c) = a-c$, 
so $a\sim c$ and the relation is transitive.
Hence, this is an equivalence relation. Write $a\equiv b\pmod{n}$ (read: $a$ is {\sl congruent} to $b\bmod{n}$) if $a\sim b$.
For any $k\in\mb{Z}$ we shall denote the equivalence class of $a$ by $\overline{a}$ -- this is called the {\sl congruent class}
or {\sl residue class} of $a\bmod{n}$ and consists of the integers which differ from $a$ by an integral multiple of $n$, i.e.,
\begin{align*}
    \overline{a} &= \{a+kn\ |\ k\in\mb{Z}\}\\
    &= \{a, a\pm n, a\pm 2n, a\pm 3n, \dots\}
\end{align*}
There are precisely $n$ distinct equivalence classes mod $n$, namely
\begin{align*}
    \overline{0}, \overline{1}, \overline{2}, \dots, \overline{n-1}
\end{align*}
\newpage
\thispagestyle{evenpagestyle}
determined by the possible remainders after division by $n$, and this residue classes
partition the integers $\mb{Z}$. the set of equivalence classes under this equivalence relation
will be denoted by $\mb{Z}/n\mb{Z}$, and called the {\sl integers modulo $n$} (or the {\sl integers mod $n$}).

The process of finding the equivalence class mod $n$ of some integer $a$ is often
referred to as {\sl reducing $a$ mod $n$}. This terminology also frequently refers to finding the smallest
nonnegative integer congruent to $a\bmod{n}$ (the {\sl least residue} of $a\bmod n$).

\Definition{0.3-1}
{We can define an addition and a multiplication for the elements of $\mb{Z}/n\mb{Z}$,
defining {\sl modular arithmetic} as follows: for $\overline{a},\overline{b}\in\mb{Z}/n\mb{Z}$,
define their sum and product by
\begin{align*}
    \overline{a} + \overline{b} = \overline{a+b}\qquad\text{and}\qquad
    \overline{a} \cdot \overline{b} = \overline{ab}
\end{align*}}
Given any two elements $\overline{a}$ and $\overline{b}$ in $\mb{Z}/n\mb{Z}$, to compute
their sum (respectively, their product) take {\sl any representative} integer $a$ in the
{\sl class} $\overline{a}$ and {\sl any representative} integer $b$ in the {\sl class} $\overline{b}$,
and add (respectively, multiply) the integers $a$ and $b$ as usual in $\mb{Z}$, and then take
the equivalence class containing the result.
\Theorem{0.3-1}
{The operations of addition and multiplication on $\mb{Z}/n\mb{Z}$ defined in \textbf{\textsf{\color{green!60!blue} Definition 0.3-1}}
are both well defined, that is, they do not depend on the choices of representatives for the classes involved.
More precisely, if $a_{1}, a_{2}\in \mb{Z}$ and $b_{1}, b_{2}\in \mb{Z}$ with $\overline{a_{1}} = \overline{b_{1}}$
and $\overline{a_{2}} = \overline{b_{2}}$, then $\overline{a_{1}+a_{2}} = \overline{b_{1} + b_{2}}$ and $\overline{a_{1}a_{2}} = \overline{b_{1}b_{2}}$, i.e., if
\begin{align*}
    a_{1} \equiv b_{1}\pmod{n}
    \qquad\text{and}\qquad
    a_{2} \equiv b_{2}\pmod{n}
\end{align*}
then
\begin{align*}
    a_{1} + a_{2} \equiv b_{1} + b_{2}\pmod{n}
    \qquad\text{and}\qquad
    a_{1}a_{2} \equiv b_{1}b_{2}\pmod{n}
\end{align*}
}
{Suppose $a_{1}\equiv b_{1}\pmod{n}$, i.e., $a_{1}-b_{1}$ is divisible by $n$.
Then $a_{1} = b_{1} + sn$ for some integer $s$.
Similarly, $a_{2}\equiv b_{2}\pmod{n}$ means $a_{2} = b_{2} + tn$ for some integer $t$.
Then $a_{1} + a_{2} = (b_{1} + b_{2}) + (s+t)n$, so $a_{1} + a_{2} \equiv b_{1} + b_{2}\pmod{n}$,
which shows that the sum of the residue classes is independent of the representatives chosen.

Similarly, $a_{1}a_{2} = (b_{1} + sn)(b_{2} + tn) = b_{1}b_{2} + (b_{1}t + b_{2}s + stn)n$,
so $a_{1}a_{2} \equiv b_{1}b_{2}\pmod{n}$, and so the product of the residue classes is also independent of the representatives chosen.
}
\Example{0.3-1}
{Find the last two digits in the number $2^{1000}$.
}
{First observe that the last two digits give the remainder of $2^{1000}$ after we divided by $100$,
so we are interested in the residue class mod $100$ containing $2^{1000}$.
We compute $2^{10} = 1024 \equiv 24\pmod{100}$, so then $2^{20} = (2^{10})^{2}\equiv 24^{2} = 576\equiv 76\pmod{100}$.
Then $2^{40} = (2^{20})^{2}\equiv 76^{2} = 5776\equiv 76\pmod{100}$.
Similarly, $2^{80} \equiv 2^{160} \equiv 2^{320} \equiv 2^{640} \equiv 76\pmod{100}$.
Finally, $2^{1000} = 2^{640}\cdot 2^{320}\cdot 2^{40} \equiv 76\cdot 76\cdot 76 \equiv 76 \pmod{100}$.
Thus, the last two digits of $2^{1000}$ are $76$.
}
\newpage
\thispagestyle{oddpagestyle}
An important subset of $\mb{Z}/n\mb{Z}$ consists of the collection of residue classes which have a multiplicative inverse in $\mb{Z}/n\mb{Z}$:
\begin{align*}
    (\mb{Z}/n\mb{Z})^{\times} = \{\overline{a}\in\mb{Z}/n\mb{Z}\ |\ \text{there exists } \overline{c}\in\mb{Z}/n\mb{Z} \text{ with } \overline{a}\cdot\overline{c} = \overline{1}\}
\end{align*}
\Proposition{0.3-1}
{\begin{align*}
    (\mb{Z}/n\mb{Z})^{\times} = \{\overline{a}\in\mb{Z}/n\mb{Z}\ |\ (a,n) = 1\}
\end{align*}}
{It is easy to see that if any representative of $\overline{a}$ is relatively prime to $n$,
then all representatives are relatively prime to $n$, so that the set on the right in the proposition is well defined.
}
If $a$ is an integer relatively prime to $n$, then the Euclidean Algorithm produces integers $x$ and $y$ satisfying
$ax+ny=1$, hence $ax\equiv 1\pmod{n}$, so that $\overline{x}$ is the multiplicative inverse of $\overline{a}$ in $\mb{Z}/n\mb{Z}$.
This gives an efficient method for computing multiplicative inverses in $\mb{Z}/n\mb{Z}$.
\Example{0.3-2}
{Find the multiplicative inverse of $\overline{17}$ in $\mb{Z}/60\mb{Z}$.
}
{Suppose $n=60$ and $a=17$. Applying the Euclidean Algorithm, we obtain
\begin{align*}
    60 &= (3) 17 + 9\\
    17 &= (1) 9 + 8\\
    9 &= (1) 8 + 1\\
    8 &= (8) 1
\end{align*}
so that $a$ and $n$ are relatively prime, and $(-7)17 + 2\cdot 60 = 1$.
Hence, $\overline{-7}=\overline{53}$ is the multiplicative inverse of $\overline{17}$ in $\mb{Z}/60\mb{Z}$.
}